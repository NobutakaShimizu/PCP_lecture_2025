\chapter{制約充足問題} \label{chap:CSP}
制約充足問題(Constraint Satisfaction Problem, CSP)は, 計算量理論において重要な問題の一つであり, PCP定理の証明においても中心的な役割を果たす.

\section{制約充足問題の定義}
制約充足問題とは端的に言えば連立方程式の解の存在性判定を問う判定問題である.

\begin{definition}{制約充足問題}{CSP}
\emph{制約充足問題(CSP)}とは次の要素からなる組$\varphi = (X,\Sigma,\calI,\calC)$を入力とする判定問題である:
\begin{itemize}
  \item \emph{アルファベット}: 有限集合$\Sigma$.
  \item \emph{変数列}: $X=(\varx_1,\dots,\varx_n)$.
  \item 制約の引数の列: $\calI=(I_1,\dots,I_m)$. ここで, 各$I_i$は $I_i\ne\emptyset$ かつ $I_i\subseteq [n]$を満たす.
  \item \emph{制約の列}: $\calC=(c_I)_{I\in\calI}$. ここで, 各$c_I$は$c_I \subseteq \Sigma^{\abs{I}}$を満たす.
\end{itemize}
各$\varx\in X$を\emph{変数}, 各$c_I\in\calC$を\emph{制約}という.

入力$(X,\Sigma,\calI,\calC)$は,
ある変数への割り当て$a \colon X \to \Sigma$が存在して, 任意の$I=\{i_1,\dots,i_k\}\in\calI$ (ただし$i_1<\dots<i_k$)について,
\begin{align*}
  a(I) := (a(\varx_{i_1}),\dots,a(\varx_{i_k})) \in c_I
\end{align*}
であるとき, かつその時に限りYesインスタンスである.

全ての$i\in[m]$について$\abs{I_i}\le q$であるとき, このCSPは\emph{$q$-CSP}という.

また, 固定した割り当て$a \colon X\to \Sigma$に対する$\varphi$の\emph{不満足度}を
\begin{align*}
  \UNSAT_\varphi(a) = \Pr_{I\sim\calI}\qty[a(I) \not\in c_I]
\end{align*}
とする. ここで$I\sim\calI$は$I$が$\calI$から一様ランダムに選ばれたことを意味する.
さらに, 全ての割り当てに関して不満足度の最小値
\begin{align*}
  \UNSAT(\varphi) = \min_{a\in\Sigma^X} \UNSAT_\varphi(a)
\end{align*}
を$\varphi$の\emph{不満足度}という.
\end{definition}

\begin{example}{グラフ彩色問題}{graph-coloring-problem-CSP}
  グラフ彩色問題(\cref{ex:graph-coloring-problem})は$2$-CSPである.
  実際, グラフ$G=(V,E)$に対して
  \begin{itemize}
    \item 変数列を$X=V$とする.
    \item アルファベットを$\Sigma=[k]$とする.
    \item 各制約$c_e$は各辺$e=\{u,v\}\in E$に付随していて,
    \begin{align*}
      (a_u,a_v) \in c_e \iff a_u\ne a_v
    \end{align*}
  と定義する.
  \end{itemize}
  このとき, グラフ$G$が$k$-彩色可能であることと, このCSPがYesインスタンスであることは同値である.
\end{example}


\subsection{PCPとの関係}
ランダムシード長$r=r(n)$かつクエリ数$q=O(1)$のPCP検証者は$\Sigma=\{0,1\}$の場合の$q$-CSPによって表現でき, 逆に$q$-CSPはPCP検証者として表現できる.
実際, $q$クエリのPCP検証者$V^\pi(x)$を考えよう.
証明の長さを$\abs{\pi}=\ell$とする.
入力$x$とランダムシード$s$を固定したときの$V^\pi(x;s)$が
証明中の読み込む文字のインデックスの集合を$I_s\subseteq[\ell]$とする (ここで$\abs{I_s}\le q$).
このとき, $V^\pi(x;s)$は$\binset^{I_s}$を$\binset$に写す関数を定める.
この関数を制約$c_s$とみなすことで, 検証者$V^\pi(x)$は$q$-CSPのインスタンスとして表現できる.
このとき, $q$-CSPのインスタンスの変数集合は証明$\pi$に対応する.
もし$x\in L$であるならば, 確率$1$で$V^\pi(x)=1$となるような$\pi$が存在する.
つまり, 全てのランダムシード$s$に対して$V^\pi(x;s)=1$となるような$\pi$が存在するため,
先ほど構成した$q$-CSPのインスタンスはYesインスタンスである.
逆に$x\not\in L$であるならば, 全ての$\pi$に対して確率$1/3$以上で$V^\pi(x)=0$となる.
これは, $q$-CSPインスタンスに対して, 全ての割り当てを考えても, 全体の制約のうち少なくとも$1/3$の割合は充足されない(すなわち, $\UNSAT$の値が$1/3$以上となる)ことを意味する.


\begin{table}[htbp]
  \centering
  \begin{tabular}{|c|c|}
    \hline
    PCP検証者 & $q$-CSP \\
    \hline
    PCP $\pi$ & 割り当て \\
    \hline
    ランダムネスを固定した時の判定 & CSPの制約 \\
    \hline
    PCP$\pi$を拒否する確率 & 割り当ての不満足度 $\UNSAT(\pi)$ \\
    \hline
  \end{tabular}
  \caption{PCPとCSPの対応関係}
  \label{table:pcp-csp-correspondence}
\end{table}

この対応関係に基づいて, PCP定理をCSPを用いた言葉で表すことができる.

\begin{theorem}{PCP定理のCSP版}{PCP-CSP-theorem}
  ある関数$m=n^{O(1)}$, $q=O(1)$, $\ell=n^{O(1)}$, 定数$c\in\Nat$, $\epsilon>0$, および
  次の性質を満たす多項式時間決定的アルゴリズム$A$が存在する:
  3彩色問題のインスタンス$G=(V,E)$を入力として受け取り, $G$の頂点数を$n$としたとき, $A$は高々$\ell(n)$個の変数と$m(n)$個の制約および要素数$c$のアルファベットからなる$q$-CSPのインスタンス$\varphi$を出力する.
  さらにこのインスタンス$\varphi$は
  \begin{itemize}
  \item 入力$G$が$\ThreeCOL$のYesインスタンスであるとき, $\UNSAT(\varphi) = 0$となる (すなわち$\varphi$はYesインスタンス).
  \item 入力$G$が$\ThreeCOL$のNoインスタンスであるとき, $\UNSAT(\varphi) \ge \epsilon$となる.
  \end{itemize}
\end{theorem}

\begin{lemma}{}{}
  \cref{thm:PCP-CSP-theorem}と\cref{thm:3-coloring-problem-PCP-verifier}は同値である.
\end{lemma}
\begin{proof}
  それぞれの方向を別々に証明する.

  \emph{\cref{thm:PCP-CSP-theorem}$\Rightarrow$\cref{thm:3-coloring-problem-PCP-verifier}の証明.}
  ある$r=O(\log n)$, $q'=O(1)$に対して, $\ThreeCOL$に対するシード長$r$, クエリ回数$q'$のPCP検証者$V^\pi$を構成する.
  入力としてグラフ$G=(V,E)$を受け取り, \cref{thm:PCP-CSP-theorem}のアルゴリズム$A$を用いて$q$-CSPのインスタンス$\varphi$を出力する.
  また, PCP $\pi$ はこの$q$-CSPのインスタンス$\varphi$の割り当てとして解釈し,
  PCP検証者$V^\pi$は以下の操作を十分大きな定数回繰り返す: 一様ランダムに制約$c_i$を選択し, その制約に含まれる変数に対する割り当て$\pi$の値を読み込み, この制約が充足\emph{されない}ならば$0$を出力し終了する. 何度も繰り返した末に終了しなかったのであれば, $1$を出力して終了する.
  この検証者は繰り返しの回数が$O(1)$であり, それぞれの繰り返しにおいては高々$q$個の変数の値を読み込むため, クエリ回数は$q'=O(q)=O(1)$となる. なお, ここでアルファベットサイズ$c$が定数であることに留意する (実際にはPCP $\pi$ は二進文字列なので, 変数割り当てを読み込む際には$\log_2 c$文字を読み込んでいる).
  また, ランダムシードはランダムな制約を選ぶために使われてるため, その長さは$O(\log m)=O(\log n)$となる.

  もしグラフ$G$がYesインスタンスならば, $\varphi$もYesインスタンスであるため, $\pi$をその充足割り当てとすれば, 全ての制約$c_i$が充足されるため, (制約の選び方のランダムネスに関して)確率$1$で$V^\pi(G)=1$となる.
  もしグラフ$G$がNoインスタンスならば, $\UNSAT(\varphi) \ge \epsilon$である. 従って, 任意の割り当て$\pi$に対して, 一様ランダムな制約$c_i$が充足される確率は高々$1-\epsilon$である.
  よって, この操作を$\ceil{10/\epsilon}=O(1)$回繰り返すと, 少なくとも確率$1/3$で充足されない制約が一度以上選ばれ, 検証者は$0$を出力する.


  \emph{\cref{thm:3-coloring-problem-PCP-verifier}$\Rightarrow$\cref{thm:PCP-CSP-theorem}の証明.}
  仮定より, $\ThreeCOL$に対する, ランダムシード長$r=O(\log n)$, クエリ回数$q=O(1)$のPCP検証者$V^\pi$が存在する. アルゴリズム$A$は, 入力$G$に対して, 全てのランダムシード$s\in \binset^r$を列挙して, それぞれの$V^\pi(G;s)$を関数$c_s$とみなして, これらを制約とするCSPを出力する. 各$V^\pi(G;s)$は, $\pi$を変数とみなしたとき, 高々$q$個の変数の値を読み込むため, $(c_s)_{s\in\binset^r}$は$2^r=n^{O(1)}$個の制約からなる$q$-CSPとなる.
  なお, $V^\pi$は多項式時間アルゴリズムなので, $A$も多項式時間アルゴリズムである.
\end{proof}

従って, 以降は\cref{thm:PCP-CSP-theorem}の証明に注力する.

\subsection{制約グラフ}
PCP定理の証明は, NP-完全な2-CSPである3彩色問題のインスタンスからスタートし,
このインスタンスを適当な$q$-CSPにうまく変換することによって与えられる.
この変換の記述を容易にするために, 2-CSPのインスタンスをグラフとして表現する方法として, \emph{制約グラフ}の概念を導入する.

\begin{definition}{制約グラフ}{constraint-graph}
  2-CSPのインスタンス$\varphi=(X,\Sigma,\calI,\calC)$に対し, 以下で定まる組$G=\ip{(V,E),\Sigma,\calC'}$を\emph{制約グラフ}という:
  \begin{itemize}
    \item $(V,E)$はグラフである. ただし頂点集合は$V=X$であり, 辺集合は$E=\calI$である. なお, ここで考えるグラフは多重辺や自己ループを持ちうるもの\footnote{具体的には$E$は多重集合であり, 自己ループに対応する辺は$\{u\}$と表す.}とし, 特に$\abs{I_i}=1$の場合は対応する辺は自己ループとする.
    \item アルファベット$\Sigma$.
    \item 制約の列$\calC'=(c'_e)_{e\in E}$は, $\abs{e}=2$ならば$c'_e=c_e$とし, $\abs{e}=1$ならば$c'_e=\qty{ (a,a) \in \Sigma^2 \colon (a)\in c_e }$とする. これにより, 全ての$c'_e$は$\Sigma^2$の部分集合となる.
  \end{itemize}

  制約グラフ$G=\ip{(V,E),\Sigma,\calC}$および割り当て$a\colon V\to \Sigma$に対して, その\emph{不満足度}を
  \begin{align*}
    \UNSAT_G(a) = \Pr_{e\sim E}[a(e) \not\in c_e]
  \end{align*}
  と定義し, $G$の不満足度を
  \begin{align*}
    \UNSAT(G) = \min_{a\in\Sigma^V} \UNSAT_G(a)
  \end{align*}
  と定義する.

  制約グラフ$G=\ip{(V,E),\Sigma,\calC}$に対して, その\emph{サイズ}を
  \begin{align*}
    \size(G) = \abs{V} + \abs{E}
  \end{align*}
  と定義する.
\end{definition}

\begin{remark}{入力長とサイズの関係}{input-length-and-size-relation}
  本講義で考える制約グラフのアルファベットサイズ$\abs{\Sigma}$は常に定数, すなわち$\abs{V}$や$\abs{E}$に依存しない値であるとする.
  この仮定の下, 制約グラフ$G=\ip{(V,E),\Sigma,\calC}$を指定するために必要なビット数を考える.
  グラフ$(V,E)$は隣接行列で表現すると$\abs{V}^2$ビットで表現できる.
  各辺$e\in E$に付随する制約$c'_e\subseteq\Sigma^2$は, 各$(a,b)\in\Sigma^2$について$(a,b)\in c'_e$かどうかを表すビットを$(a,b)$について並べれば良いため, 全部で$\abs{E}\cdot\abs{\Sigma}^2$ビットで表現できる.
  よって, 制約グラフ$G$は$\abs{V}^2+\abs{E}\cdot\abs{\Sigma}^2 = O(\size(G)^2)$ビットで表現できる.
  このことから, 制約グラフを入力として受け取るアルゴリズムが多項式時間かどうかを議論する際は, その時間計算量が$\size(G)$に関して多項式かどうかを議論すれば良いことになる.
\end{remark}

\begin{remark}{制約グラフの性質}{constraint-graph-properties}
  制約グラフ$G=\ip{(V,E),\Sigma,\calC}$に対して, その構造を表すグラフ$(V,E)$の性質を表す用語はそのまま制約グラフにも適用する.
  例えば, $(V,E)$が連結であるときに$G$は連結であるといい, $(V,E)$が正則であるならば$G$は正則であるという.
\end{remark}


\section{エクスパンダーグラフ}
\subsection{エクスパンダーグラフの定義}
\subsection{エクスパンダー混交補題}
\subsection{制約グラフのエクスパンダー化}