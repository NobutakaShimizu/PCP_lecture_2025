\chapter{制約充足問題} \label{chap:CSP}
制約充足問題(Constraint Satisfaction Problem, CSP)は, 計算量理論において重要な問題の一つであり,
PCP定理の証明においても中心的な役割を果たす.

\section{制約充足問題の定義}
制約充足問題とは端的に言えば抽象化された連立方程式である.
複数の変数$x_1,\dots,x_n$があり, それぞれの変数は有限集合$\Sigma$に値をとる.
そしてこれらの変数が満たすべき複数の制約が与えられたとき, それらの制約を全て同時に満たす$(x_1,\dots,x_n)$の割り当てを求める問題を制約充足問題という.

\begin{definition}{制約充足問題}{CSP}
\emph{制約充足問題(CSP)}とは次の要素からなる組$(X,\Sigma,\calI,\calC)$を入力とする判定問題である:
\begin{itemize}
  \item \emph{アルファベット}と呼ばれる有限集合$\Sigma$.
  \item 変数集合$X=\{x_1,\dots,x_n\}$.
  \item 部分集合族$\calI=\{I_1,\dots,I_m\}$. ただし各$I_i$は$I_i\subseteq[n]$である.
  \item 関数族$\calC=\{c_1,\dots,c_m\}$. ただし各$c_i$は$c_i: \Sigma^{I_i}\to\{0,1\}$である.
\end{itemize}
入力$(X,\Sigma,\calI,\calC)$は,
ある変数への割り当て$(a_1,\dots,a_n)\in\Sigma^n$が存在して, 任意の$i\in[m]$について$c_i(a_{I_i})=1$であるとき, かつその時に限りYesインスタンスである.

また, 全ての$i\in[m]$について$\abs{I_i}\le k$であるとき, このCSPは\emph{$k$-CSP}と呼ばれる.
\end{definition}

\begin{example}{グラフ彩色問題}{graph-coloring-problem-CSP}
  グラフ彩色問題(\cref{ex:graph-coloring-problem})は$2$-CSPである.
  実際, グラフ$G=(V,E)$に対して
  \begin{itemize}
    \item 変数集合を$X=V$とする.
    \item アルファベットを$\Sigma=[k]$とする.
    \item 制約集合を$\calI=\{E\}$とする.
    \item 関数族を$\calC=\{c_e\}_{e\in E}$とし, 各$e=\{u,v\}\in E$に対して
    \begin{align*}
      c_e(a_u,a_v) = \begin{cases}
        1 & \text{if } a_u\neq a_v, \\
        0 & \text{if } a_u=a_v
      \end{cases}
    \end{align*}
    と定義する.
  \end{itemize}
  このとき, グラフ$G$が$k$-彩色可能であることと, このCSPがYesインスタンスであることは同値である.
\end{example}

\subsection{PCPとの関係}
\subsection{制約グラフ}

\section{PCP定理の証明の概要}

\section{エクスパンダーグラフ}
\subsection{エクスパンダーグラフの定義}
\subsection{エクスパンダー混交補題}
\subsection{制約グラフのエクスパンダー化}