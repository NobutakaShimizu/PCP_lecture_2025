\newcommand{\kara}{} %無を出力するコマンド

%def
\newtcolorbox[auto counter, number within=section, crefname = {定義}{定義}]{definition}[3][]
{enhanced, breakable = true, fonttitle = \bfseries,
title = 定義~\thetcbcounter~\if #2\kara \else(#2) \fi,
#1,
label = def:#3}

%thm
\newtcolorbox[auto counter, use counter from=definition, crefname = {定理}{定理}]{theorem}[3][]
{enhanced, colback = orange!10!white, colframe = red!50!black, breakable = true, fonttitle = \bfseries,
title = 定理~\thetcbcounter~\if #2\kara \else(#2) \fi,
#1,
label = thm:#3}

%prop
\newtcolorbox[auto counter, use counter from=definition, crefname = {命題}{命題}]{proposition}[3][]
{enhanced,  colback = orange!10!white, colframe = red!50!black, breakable = true, fonttitle = \bfseries,
title = 命題~\thetcbcounter~\if #2\kara \else(#2) \fi,
#1,
label = prop:#3}

%lemma
\newtcolorbox[auto counter, use counter from=definition, crefname = {補題}{補題}]{lemma}[3][]
{enhanced,  colback = green!10!white, colframe = green!50!black, breakable = true, fonttitle = \bfseries,
title = 補題~\thetcbcounter~\if #2\kara \else(#2) \fi,
#1,
label = lem:#3}

%cor
\newtcolorbox[auto counter, use counter from=definition, crefname = {系}{系}]{corollary}[3][]
{enhanced, breakable = true, fonttitle = \bfseries,
title = 系~\thetcbcounter~\if #2\kara \else(#2) \fi,
#1,
label = cor:#3}

%remark
\newtcolorbox[auto counter, use counter from=definition, crefname = {注釈}{注釈}]{remark}[3][]
{enhanced, breakable = true, colback = white, fonttitle = \bfseries,
title = 注釈~\thetcbcounter~\if #2\kara \else(#2) \fi,
#1,
label = rem:#3}

%exercise
\newtcolorbox[auto counter, crefname = {演習問題}{演習問題}]{exercise}[3][]
{enhanced, breakable = true, colback = blue!10!white, colframe = blue!50!black, fonttitle = \bfseries,
title = 演習問題~\thetcbcounter~\if #2\kara \else(#2) \fi,
#1,
label = exer:#3}

%example
\newtcolorbox[auto counter, use counter from=definition, crefname = {例}{例}]{example}[3][]
{enhanced, breakable = true, colback = brown!10!white, colframe = brown!50!black, fonttitle = \bfseries,
title = 例~\thetcbcounter~\if #2\kara \else(#2) \fi,
#1,
label = ex:#3}

%algorithm
\newtcolorbox[auto counter, use counter from=definition, crefname = {アルゴリズム}{アルゴリズム}]{algorithm}[3][]
{enhanced, breakable = true, colback = cyan!10!white, colframe = cyan!50!black, fonttitle = \bfseries,
title = アルゴリズム~\thetcbcounter~\if #2\kara \else(#2) \fi,
#1,
label = alg:#3}

%claim
\newtcolorbox[auto counter, use counter from=definition, crefname = {主張}{主張}]{claim}[3][]
{enhanced, breakable = true, colback = yellow!10!white, colframe = yellow!50!black, fonttitle = \bfseries,
title = 主張~\thetcbcounter~\if #2\kara \else(#2) \fi,
#1,
label = claim:#3}

\renewcommand{\proofname}{\textbf{証明}}
\renewcommand{\figurename}{図}
\crefname{equation}{式}{式}
\crefname{section}{セクション}{セクション}
\crefname{figure}{図}{図}