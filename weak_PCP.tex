\chapter{局所的な検証と弱いPCP定理の証明}

この章はPCP検証者における「少ないクエリ回数」という要件がなぜ達成できるかについて説明することを目的とする.
そのため, まずはPCP定理の要件であるランダムビットの短さはひとまず無視した以下の形の「弱いPCP定理」を証明する: 任意の$\NP$に属する判定問題が$O(1)$クエリかつ$\poly(n)$ビットのランダムネスを用いて確率的に検証可能である.

\section{局所的な検証}
本講義では, 検証者が証明の文字列$\pi$のうち$O(1)$個の文字を読み込んで検証を行うことを\emph{局所的な検証}と呼ぶことにする.
一般的な感覚として, ある主張が成り立つことを示す証明を検証する際, \emph{冗長に表現で記述しない限り}その証明を全て読み込まなければ検証できないと思われる (全ての文字に本質的な意味があるならば全てを確認しなければならないはずであろう).
では逆に, 証明を冗長に表現することで局所的な検証を可能にすることはできるだろうか?

\subsection{線形方程式の局所的な検証(1/2)}
興味深いことに, 非常に冗長性が大きい記述で証明が与えられたならば局所検証が可能であることを, 線形方程式の検証を例に説明する.

\begin{definition}{線形方程式}{linear-equation}
有限体$\F$上の行列$M\in\F^{m\times n}$とベクトル$z\in\F^m$に対して
\begin{align*}
  My = z
\end{align*}
を満たす$y\in\F^n$が存在するかどうかを判定する問題を$\LinEq$とする.
ここで$\abs{\F}=q$は$m,n$とは独立な定数であるとする (従って有限体上の演算は$O(1)$時間で計算できると仮定する).
\end{definition}

愚直な検証者として$y\in\F^n$を証拠として受け取り, $My=z$を確認することで検証を行うことを考える (実際の計算機では全てを二進文字列で表現するため, $\F$の元は$\ceil{\log_2\abs{\F}}=O(1)$ビットで表現されている).
明らかにこの検証者は$My$を計算するために$y$の全ての成分 (すなわち$O(n)$ビット) を読み込む必要があるため, 局所的な検証者とはならない.
もちろん, この判定問題は標準的な線型方程式の解の存在性判定であるため, 掃き出し法を用いれば証拠へのオラクルアクセスなしでも多項式時間で解ける.
しかしここではあえて, 局所的な検証の構成例を与えるという目的で取り扱う.

重要な性質として, 二つのベクトル$a,b\in\F^n$に対する次の性質を考える (証明は\cref{exer:random-inner-product}):
\begin{align}
  \Pr_{r\sim \F^n} \qty[ r^\top a \ne r^\top b ] = \begin{cases}
    0 & \text{if } a=b \\
    1-\frac{1}{q} & \text{if } a\ne b.
  \end{cases} \label{eq:linear-equation-property}
\end{align}
さて, \cref{eq:linear-equation-property}の性質を用いると一様ランダムな$r\sim\F^n$に対して
\begin{align}
  r^\top M y = r^\top z \label{eq:random-inner-product}
\end{align}
かどうかを確認することで, $My=z$かどうかを確率的に検証できる.
実際, $My=z$ならば確率$1$で検証者は受理し,
$My\ne z$ならば確率$1-1/q$で検証者は拒否する (何度も繰り返せばこの拒否率を任意に$1$に近い定数まで近づけることができる).
ベクトル$z$は入力として与えられているため\cref{eq:random-inner-product}の右辺は$O(n)$時間で計算できる.
一方で, ベクトル$y$は証拠として与えられるため, $r^\top M y$の計算は愚直に考えると$O(mn)$時間かかる上に$y$の全ての成分を読み込む必要がある.
ところが, 全てのベクトル$w\in \F^n$に対して$y^\top w\in\F$を連結して得られる長さ$q^n$の文字列$(y^\top w)_{w\in \F^n}$を証拠$\pi$として与えることにより,
自身で$r^\top M$を計算した後に $r^\top M y = y^\top (M^\top r)$の値を\emph{一文字の証拠の読み込み}で計算できるのである.

\begin{exercise}{ランダムなベクトルとの内積}{random-inner-product}
  \cref{eq:linear-equation-property}を証明せよ.
\end{exercise}

以上の議論より, $\LinEq$に対する局所的な検証として以下の検証者を考えることができる:
\begin{algorithm}{$\LinEq$に対する局所的な検証?}{local-verification-LinEq?}
  \begin{enumerate}
  \item 入力として$M,z$を受け取り, 証拠として$\pi \in \F^{q^n}$へのオラクルアクセスを受け取る.
  \item 一様ランダムな$r\sim\F^m$を選択し, $r^\top M$と$r^\top z$を計算する.
  \item 証拠$\pi\colon \F^{q^n}$を関数$\pi\colon \F^n\to \F$として解釈し, オラクルアクセスを用いて$a=\pi(M^\top r)$を求める.
  \item $a=r^\top z$ならば$1$を出力し, そうでなければ$0$を出力する.
  \end{enumerate}
\end{algorithm}

上記の検証者はPCP検証者の性質を持つだろうか?
まず, $(M,z)$がYesインスタンスであるとき, ある$y\in\F^n$が存在して$My=z$が成り立つ.
このとき, $\pi\colon\F^n\to\F$を$\pi(w)=y^\top w$とすると, $\pi(M^\top r)=y^\top (M^\top r)=r^\top (My)=r^\top z$となるため, 検証者は確率$1$で受理する.

一方で$(M,z)$がNoインスタンスであるときに検証者が拒否する確率はどうなるだろうか?
PCP検証者であることを示す(cf. \cref{def:PCP})には, 任意の$\pi\in\F^n\to\F$に対して拒否確率は少なくとも$1/3$以上でなければならない.
これは示せるのであろうか?
ある$y\in\F^n$に対して$\pi(w)=y^\top w$と表せるような$\pi\in\F^n\to\F$に対しては, \cref{eq:random-inner-product}の性質および$(M,z)$がNoインスタンスであることから$My\ne z$より, 少なくとも確率$1-1/q$で検証者は拒否する.
しかしながら, 一般の$\pi\in\F^n\to\F$はこのような線形関数として表現できるとは限らない.

\subsection{線形性テスト}
$\LinEq$に対するPCP検証者を構成するために,
\cref{alg:local-verification-LinEq?}を以下のように修正する:
オラクル$\pi$が, ある$y\in\F^n$に対して$\pi(w)=y^\top w$と表せるならば, \cref{alg:local-verification-LinEq?}のステップ2-4を実行し, そうでないならば$0$を出力する.
ここで新たに一つの問題が生じる: どのようにして$\pi$の線形性を局所検証すればよいだろうか?

一般にオラクルアクセスで与えられた関数$\pi\colon\F^n\to\F$が線形関数かどうかを厳密に確認するには,
\begin{align}
  \forall x,y\in\F^n,\quad \pi(x+y)=\pi(x)+\pi(y) \label{eq:linearity-test-all-points}
\end{align}
が成り立つかどうかを確認する必要があり, $q^n$回のオラクルアクセスが必要となってしまう.

そこで, 「線形関数であるかどうか」ではなく「線形関数に近いかどうか」を局所検証することを考える.
\begin{definition}{関数同士の近さ}{distance-between-functions}
  二つの関数$f,g\colon A \to B$に対して$\dist(f,g)$を
  \begin{align*}
    \dist(f,g) =\Pr_{a\sim A} \qty[ f(a) \ne g(a) ]
  \end{align*}
  と定義し, $\dist(f,g)\le\delta$を満たすとき, $f$は$g$に\emph{$\delta$-近い}($\delta$-close)という.
  
  また, 関数クラス$\calF\subseteq \qty{ g\colon A\to B }$および関数$f\colon A\to B$に対して
  \begin{align*}
    \dist(f,\calF) = \min_{g\in\calF} \dist(f,g)
  \end{align*}
  と定義し, $\delta(f,\calF)\le \delta$であるとき, $f$は$\calF$に$\delta$-近いという.
\end{definition}

関数$f\colon A\to B$をベクトル$f\in B^A$と同一視すると, $\dist(f,g)$はベクトル$f,g$間の正規化されたハミング距離($\ell_0$ノルム)に一致する.

関数クラス$\calF\subseteq \qty{ \rho\colon \F^n\to\F }$を線形関数全体, すなわち
\begin{align}
  \calF = \qty{ \rho\colon \F^n\ni x\mapsto y^\top x\in \F \colon y\in \F^n } \label{eq:linear-functions}
\end{align}
とする.
オラクルアクセスとして与えられた関数$\pi\colon \F^n\to\F$が$\calF$に近いかどうかを検証する局所検証者が構成できる \citet{BLR93}.

\begin{theorem}{線形性テスト}{linearity-test}
  関数クラス$\calF$を\cref{eq:linear-functions}で定義する.
  以下を満たす乱択多項式時間オラクルアルゴリズム$A^\pi(1^n)$が存在する\footnote{アルゴリズム$A$は$1^n$を入力として与えられているため, 多項式時間であることは計算量が$\poly(n)$で抑えられることを意味する.}:
  関数$\pi\colon \F^n\to\F$がオラクルアクセスとして与えられたとき,
  \begin{itemize}
    \item $\pi\in \calF$ならば$A^\pi(1^n)$は確率$1$で受理する.
    \item $\dist(\pi,\calF)\ge 0.01$ならば$A^\pi(1^n)$は確率$0.99$で拒否する.
  \end{itemize}
\end{theorem}
\begin{proof}
  アルゴリズム$A^\pi$は非常に単純で, 線形関数であることの特徴づけ\cref{eq:linearity-test-all-points}をランダムな点で確認するというものである.
  具体的には以下で与えられる:
  \begin{algorithm}{線形性テスト}{linearity-test}
    \begin{enumerate}
      \item オラクルアクセスとして関数$\pi\colon \F^n\to\F$を受け取り, 入力として$1^n$を受け取る.
      \item 一様ランダムに$x,y\sim\F^n$を選び, $\pi(x+y)\ne \pi(x)+\pi(y)$が成り立つならば拒否する.
      \item 十分大きな定数$K\in\Nat$に対し, ステップ2を$K$回繰り返す. この繰り返しの中で一度も拒否しなければ, 受理する.
    \end{enumerate}
  \end{algorithm}

  実際, $\pi\in\calF$ならば\cref{eq:linearity-test-all-points}が成り立つため, $A^\pi(1^n)$は確率$1$で受理する.
  一方で, $\dist(\pi,\calF)\ge 0.01$であるときに拒否確率を下から抑えたい.
  これは以下の主張から従う:

  \begin{claim}{線形性テストの拒否確率}{linearity-test-rejection-probability}
    関数クラス$\calF$を\cref{eq:linear-functions}で定義する.
    任意の関数$\pi\colon \F^n\to\F$に対して
    \begin{align*}
      \Pr_{x,y\sim\F^n}\qty[ \pi(x+y) \ne \pi(x)+\pi(y) ] \ge \frac{1}{6}\cdot \dist(\pi,\calF).
    \end{align*}  
  \end{claim}

  \cref{claim:linearity-test-rejection-probability}は後で証明する.
  仮定より$\dist(\pi,\calF)\ge \delta:=0.01$なので,
  ステップ3の定数$K$を十分大きくすることによって, \cref{alg:linearity-test}の拒否確率を$0.99$より大きくすることができる.
  \end{proof}

  次に\cref{claim:linearity-test-rejection-probability}を示す.
  実際にはより一般的な次の補題を証明する.
  \begin{lemma}{準同型性テスト}{homomorphism-test}
    有限アーベル群$G,H$に対し, 写像$f\colon G\to H$が準同型であるとは, 任意の$x,y\in G$に対して$f(x+y)=f(x)+f(y)$が成り立つことをいい, $G$から$H$への準同型な写像の全体を$\calH$と表す.
    このとき, 任意の関数$f\colon G\to H$に対して
    \begin{align*}
      \Pr_{x,y\sim G}\qty[ f(x+y)\ne f(x)+f(y) ] \ge \frac{1}{12}\cdot \dist(f,\calH).
    \end{align*}
  \end{lemma}
  \begin{proof}
    記号の簡単のため, $\rho_f:=\Pr_{x,y\sim G}\qty[ f(x+y)\ne f(x)+f(y) ]$と表す.
    また, 各$y\in G$に対し, 写像$v_y\colon G\to H$を$v_y(x)=f(x+y)-f(y)$とし, $v\colon G\to H$を
    \begin{align*}
      v(x) = \mathrm{argmax}_{a \in H} \qty{ \Pr_{y\sim G}\qty[ v_y(x)=a ] }
    \end{align*}
    で定める (タイが存在する場合は任意).
    すなわち, $v(x)$は$(v_y(x))_{y\in G}$の中での多数決, すなわち最も出現頻度の高い値として定める.
    証明は以下の三つの主張を組み合わせることによって得られる.
    
    \begin{claim}{}{claim1}
      $\dist(f,v) \le 2\rho_f$が成り立つ.
    \end{claim}
    \begin{proof}
      定義より$\rho_f = \Pr_{x,y}[f(x) + f(y) \ne f(x+y)] = \Pr_{x,y} [ f(x) \ne v_y(x)]$ である.
      また, $v(x)\ne h\Rightarrow \Pr_y[v_y(x)\ne h]\le 1/2$が成り立つ.
      実際, $(v_y(x))_{y\in G}$の中で多数決をとったときに$h$が選ばれなかったということは, 過半数に至らなかったことを意味するからである.
      特に, $h=f(x)$とすると, $\mathbf{1}_{v(x)\ne f(x)} \le 2\Pr_y[v_y(x)\ne f(x)]$.
      両辺の$x\sim G$に関する期待値をとると
      \begin{align*}
        \dist(f,v) = \Pr_x[v(x)\ne f(x)] \le 2\Pr_{x,y}[v_y(x)\ne f(x)] =2\rho_f.
      \end{align*}
      となり主張を得る.
    \end{proof}
    
    \begin{claim}{}{claim2}
      $\rho_f<1/6$ ならば, 全ての$x\in G$に対して$\Pr_y[v_y(x)= v(x)] > 2/3$.
    \end{claim}
    \begin{proof}
      任意に $x\in G$ を固定し, 確率変数$\Phi$を, 一様ランダムに$y\sim G$を選び $\Phi=v_y(x)$ として定める. 我々の目標は, ある$a\in H$に対して$\Pr[\Phi=a]>2/3$ が成り立つことを示すことである.
      $\Phi_1,\Phi_2$を$\Phi$の独立なコピーとする (固定する$x$は同一). $\Phi$がある値をとる傾向にあることを示すために, まずは $\Phi_1=\Phi_2$ が成り立つ確率が大きいことを示す.
      \begin{align*}
      \Pr[\Phi_1 = \Phi_2] &= \Pr_{y_1,y_2\sim G}[ v_{y_1}(x) = v_{y_2}(x) ] \\
      &= \Pr_{y_1,y_2}[ f(x+y_1) - f(y_1) = f(x+y_2) - f(y_2) ] \\
      &= \Pr_{y_1,y_2}[ f(x+y_1)-f(x+y_2) = f(y_1)-f(y_2) ] \\
      &\ge 1-2\rho_f \\
      &> 2/3.
      \end{align*}    
      最後の不等式では, $\Pr_{y_1,y_2}[f(x+y_1)-f(x+y_2) \ne f(y_1-y_2)]\le \rho_f$ および $\Pr_{y_1,y_2}[f(y_1)-f(y_2)\ne f(y_1-y_2)]\le \rho_f$ を用いた.    
      ここで, $2/3<\Pr[\Phi_1=\Phi_2]=\sum_{a\in H}\Pr[\Phi=a]^2 \le \max_a\Pr[\Phi=a]\cdot \sum_{a\in H}\Pr[\Phi=a] = \max_a\Pr[\Phi=a]$ より主張を得る.
    \end{proof}

    \begin{claim}{}{claim3}
      全ての $x\in G$ に対して $\Pr_y[v_y(x)=v(x)]>2/3$ が成り立つならば, $v\in \calH$, すなわち$h$は準同型である.
    \end{claim}
    \begin{proof}
      全ての$x,y\in G$に対して$f(x)+f(y)=f(x+y)$が成り立つことを示せばよい. 任意に$x,y\in G$を固定する.
      一様ランダムな$z\sim G$ を選ぶ. このとき, 仮定より
      \begin{itemize}
        \item $\Pr_z[v_z(x)\ne v(x)]<1/3$,
        \item $\Pr_z[v_{z-y}(y)\ne v(y)]<1/3$,
        \item $\Pr_z[v_{z-y}(x+y)\ne v(x+y)]<1/3$
      \end{itemize}
      従って, ユニオンバウンドより, 任意の $x,y\in G$ に対してある$z\in G$が存在して次の三つが同時に成り立つ:
      \begin{itemize}
        \item $v(x)=v_z(x)=f(x+z)-f(z)$,
        \item $v(y)=v_{z-y}(y)=f(z)-f(z-y)$,
        \item $v(x+y)=v_{z-y}(x+y)=f(x+z)-f(z-y)$
      \end{itemize}
      これらの等式から
      \begin{align*}
        v(x)+v(y)-v(x+y) = f(x+z)-f(z-y) - (f(x+z)-f(z-y)) = 0.
      \end{align*}
      より主張を得る.
    \end{proof}
    最後に\cref{lem:homomorphism-test}の証明を完成させる.
    $\rho_f<1/6$ならば, \cref{claim:claim2,claim:claim3}より, $v\in\calH$である.
    さらに\cref{claim:claim1}より,
    \begin{align*}
      \rho_f \ge \frac{\dist(f,v)}{2} \ge \frac{\dist(f,\calH)}{2}
    \end{align*}
    である. 一方, そうでなければ$\rho_f\ge 1/6 \ge \dist(f,\calH)/6$である.
    いずれにせよ, $\rho_f\ge \dist(f,\calH)/6$が成り立つ.

  \end{proof}

\subsection{線形方程式の局所的な検証(2/2)}
ここまでの議論を用いて$\LinEq$に対する局所的な検証者を構成する.
\begin{algorithm}{$\LinEq$に対する局所的な検証}{local-verification-LinEq}
  \begin{enumerate}
    \item 入力として$M,z$を受け取り, 証拠として$\pi \in \F^{q^n}$へのオラクルアクセスを受け取る.
    \item \cref{thm:linearity-test}のアルゴリズムを$\pi$をオラクルとして実行し, $\dist(\pi,\calF)\ge 0.01$ならば拒否して終了する.
    \item 一様ランダムな$r\sim\F^m$を選択し, $r^\top M$と$r^\top z$を計算する.
    \item 新たに一様ランダムなベクトル$r'\sim\F^n$を選び, オラクルアクセスを用いて$a=\pi(M^\top r + r') - \pi(r')$を求める.
    \item $a=r^\top z$ならば受理し, そうでなければ拒否する.
  \end{enumerate}
\end{algorithm}
\cref{alg:local-verification-LinEq?}の検証者と比較すると,
まず, 証拠として与えられた関数$\pi$が線形関数であることを検証するためのステップが追加されている.
また, ステップ4のオラクルアクセスの方法が異なる.
具体的には, \cref{alg:local-verification-LinEq?}のステップ3では“$a=\pi(M^\top r)$"としているが,
\cref{alg:local-verification-LinEq}では“$a=\pi(M^\top r + r') - \pi(r')$"としている.
この修正の意味について説明する.
まず, $\pi$が線形関数であることを検証するステップが追加されているため, $\pi$に最も近い線形関数を$f\colon\F^n\to\F$とすると,
ステップ4の処理が実行される時点で$\dist(\pi,f)\le 0.01$を満たすことが確定している (このことから特に, $\pi$に最も近い線形関数は唯一存在することが保証される. 証明は\cref{exer:uniqueness-of-nearest-linear-function}).
すなわち, 一様ランダムなベクトル$r'\sim\F^n$に対して$\Pr[\pi(r')=f(r')] \ge 0.99$が成り立つ.
ここで, $r\sim\F^m$に対してベクトル$M^\top r$の分布は必ずしも一様ランダムとはならない (例えば$M=O$がゼロ行列ならば常に$M^\top r=0$である).
従って, $f(M^\top r)$は$\pi(M^\top r)$によって計算できるとは限らない.
一方, \cref{alg:local-verification-LinEq}のステップ4では, 一様ランダムなベクトル$r'\sim\F^n$を選択し, オラクルアクセスを用いて$a=\pi(M^\top r + r') - \pi(r')$を求めている.
このとき, $M^\top r + r'$と$r'$はどちらも(独立ではない)一様ランダムなベクトルとなる.
従って, $f$の線形性より, 確率$0.98$で
\begin{align*}
  f(M^\top r) = f(M^\top r + r') - f(r') = \pi(M^\top r+r') - \pi(r') = a
\end{align*}
が成り立つ.
まとめると, ステップ4は$\pi$に最も近い線形関数$f$に対して$f(M^\top r)$を計算するための操作となっている.

\begin{exercise}{最も近い線形関数の唯一性}{uniqueness-of-nearest-linear-function}
  線形関数の全体を$\calF$とする (\cref{eq:linear-functions}).
  関数$\pi\colon\F^n\to \F$が$\dist(\pi,\calF) \le 0.01$を満たすとき,
  $\dist(\pi,f)\le 0.01$を満たす$f\in\calF$は唯一存在することを示せ.
\end{exercise}

これまでの議論から, \cref{alg:local-verification-LinEq}は$\LinEq$に対する局所的な検証者であることがわかる.

\begin{theorem}{線形方程式の局所的な検証}{local-verification-LinEq}
  \cref{alg:local-verification-LinEq}の検証者を$V^\pi(M,z)$とする.
  このとき, 任意の$M\in\F^{m\times n},z\in\F^m$に対して以下が成り立つ:
  \begin{itemize}
    \item $V^\pi(M,z)$の$\pi$へのオラクルアクセスの回数は$O(1)$である.
    \item $My=z$を満たす$y\in\F^n$が存在するならば, $V^\pi(M,z)$は確率$1$で受理する.
    \item $My=z$を満たす$y\in\F^n$が存在しないならば, $V^\pi(M,z)$は確率$2/3$で拒否する.
  \end{itemize}
\end{theorem}
\begin{proof}
  オラクルアクセスの回数は明らかに$O(1)$である.
  また, $My=z$を満たす$y\in\F^n$が存在するならば, $\pi\colon\F^n\to\F$として
  \begin{align*}
    \pi(x) = x^\top y
  \end{align*}
  とすれば, $\pi$は線形関数であり, さらに
  \begin{align*}
    a = \pi(M^\top r + r') - \pi(r') = r^\top M y
  \end{align*}
  となるため, 確率$1$で受理する.

  一方, $My=z$を満たす$y\in\F^n$が存在しないときに検証者が拒否する確率を考える.
  オラクル$\pi$に関して以下の二つのケースを考える:

  \paragraph*{ケース1: $\pi$が線形関数に$0.01$-近いとき.}
  このとき, $V^\pi(M,z)$がステップ5で拒否する確率を評価する.
  オラクル$\pi$に最も近い線形関数を$f\colon \F^n\to\F$とする.
  ステップ4で選ばれた一様ランダムな$r'$に対し, $M^\top r + r$と$r'$はどちらも一様ランダムなベクトルであり, $\dist(\pi,f)\le 0.01$より,
  \begin{align*}
    &\Pr_{r'}\qty[ \pi(M^\top r + r') \ne f(M^\top r + r') ] \le 0.01,\\
    &\Pr_{r'}\qty[ \pi(r') \ne f(r') ] \le 0.01.
  \end{align*}
  従って, ユニオンバウンドより, ステップ4で計算した$a$に対して
  \begin{align}
    \Pr_{r,r'}\qty[ a = f(M^\top r) ] \ge 0.98 \label{eq:linearity-test-rejection-probability-1}
  \end{align}
  となる.
  ここで, 線形関数$f$が$f(x)=x^\top y$と表せるとすると, 拒否確率は
  \begin{align*}
    \Pr_{r,r'}\qty[ a \ne r^\top z ] &\ge \Pr_{r,r'}\qty[ a\ne r^\top z \mid a = f(M^\top r) ] \cdot \Pr_{r,r'}\qty[ a = f(M^\top r) ] \\
    &\ge \Pr_{r}\qty[ r^\top M y \ne r^\top z ]\cdot 0.98 & & \because\text{\cref{eq:linearity-test-rejection-probability-1}} \\
    &\ge \qty(1-\frac{1}{q}) \cdot 0.98 & & \because \text{\cref{eq:random-inner-product}および$My\ne z$} \\
    &\ge 2/3 & & \because\text{$q\ge 2$}
  \end{align*}
  を満たす
  

  \paragraph*{ケース2: $\pi$が線形関数に$0.01$-遠いとき.}
  このとき, \cref{thm:linearity-test}より, $V^\pi(M,z)$はステップ2において少なくとも確率$0.99$で拒否する.
  
  どちらのケースでも, $V^\pi(M,z)$は少なくとも確率$2/3$で拒否する.
  
\end{proof}


\section{弱いPCP定理とその証明}
\cref{thm:local-verification-LinEq}のアイデアを拡張してPCP定理を証明するには何が必要だろうか?
\begin{itemize}
  \item まず, $\LinEq$は実際には多項式時間で解ける判定問題であるが, 実際には何かしらのNP完全な判定問題に対して\cref{thm:local-verification-LinEq}のような検証者を構成しなければならない.  
  \item 次に, \cref{thm:local-verification-LinEq}の検証者$V^\pi(M,z)$は, ランダムに$r,r'\sim\F^n$を選ぶ時点で$\Omega(n)$ビットのランダムネスを必要とする. 実際, 証拠として与えられた関数$\pi$を文字列として表現するとその長さは$q^n$に比例するため, その文字列のランダムなインデックスを指定しようとすると必然的に$\Omega(n)$ビットのランダムネスが必要である. そこで証明$\pi$の長さを指数的な長さ$2^{\Theta(n)}$から多項式$n^{O(1)}$に抑える必要がある (このとき, インデックスの指定に必要なビット数は$O(\log n)$で抑えられる).
\end{itemize}

本節では前者の問題を解決し, 以下の弱いPCP定理を証明する.
\begin{theorem}{弱いPCP定理}{weak-PCP-theorem}
  ある$r(n)=\poly(n),q(n)=O(1)$に対して, $\NP \subseteq \PCP(r,q)$が成り立つ.
\end{theorem}

我々の最終的な目標であるPCP定理(\cref{thm:PCPtheorem})では, PCP検証者のランダムビットの長さが$r(n)=O(\log n)$であることを要求しているため, $r(n)$に関しては\cref{thm:weak-PCP-theorem}は指数的に大きい値になってしまっているが, 読み込む証明の文字数は$n$に依存しない定数で抑えられるという点では同一である.
なお, \cref{thm:weak-PCP-theorem}は後に\cref{thm:PCPtheorem}の証明でも用いられる.

\subsection{方針}
  NP完全な判定問題として3彩色問題$\ThreeCOL$を考え, ある$r=O(1),q=\poly(n)$に対して$\ThreeCOL\in\PCP(r,q)$を示す.
  すると, \cref{thm:weak-PCP-theorem}の証明は\cref{exer:exercise2}と同じ議論から得られる.

  3彩色問題$\ThreeCOL$のインスタンス$G=(V,E)$を考える.
  以降, $\F_3$を$\F$と略記する.
  これがYesインスタンスであることと, ある$x=(x(u))_{u\in V}\in\F^V$が存在して
  \begin{align}
    \forall e=\{u,v\}\in E, \quad (x(u) - x(v))^2 = 1 \label{eq:3-coloring-F3}
  \end{align}
  を満たすことは同値である. 従って, 二次方程式系\cref{eq:3-coloring-F3}に対する所望のPCP検証者を構成すればよい.
  まずはこの二次方程式系を線型方程式系として表現するために, 以下の行列$M \in \F^{E \times V^2}$を定義する:
  \begin{align*}
    M(e, (u,v)) = \begin{cases}
      1 & \text{if } u=v\in e,\\
      -2 & \text{if } e=\{u,v\},\\
      0 & \text{otherwise}.
    \end{cases}
  \end{align*}
  ベクトル$x\in\F^V$に対して$y\in\F^{V^2}$を$y(u,v)=x(u)\cdot x(v)$と定義する (特に$u=v$ならば$y(u,u)=x(u)^2$である).
  このとき, \cref{eq:3-coloring-F3}は$My=\allone$と表現できる.
  ここまでの議論から, $G=(V,E)$がYesインスタンスであることと, ある$y\in \F^{V^2}$が存在して
  \begin{enumerate}
    \item ある$x\in\F^V$が存在して, 全ての$u,v\in V^2$に対し$y(u,v)=x(u)\cdot x(v)$.
    \item $My=\allone$.
  \end{enumerate}
  を満たすことは同値である.
  方針としては, この二つの条件を検証するPCP検証者を構成して組み合わせることによって, 上記の問題に対する検証者を構成する.
  条件2に関しては, \cref{thm:local-verification-LinEq}の検証者$V^\pi(M,z)$を用いることができる.
  この検証者はオラクルアクセスの回数が$O(1)$でランダムビット長が$O(\abs{V^2})=O(n^2)$である.
  
  \subsection{テンソル積の確認}
  条件1に対する検証者を与える.
  ベクトル$y\in\F^{V^2}$に対し, それを$V\times V$行列とみなしたものを$Y=(y(u,v))_{u,v\in V}$とする.
  このとき, ある$x\in\F^V$が存在して$Y=x x^\top$となることを検証すればよい.
  以後, $X=xx^\top$と略記する.
  
  \cref{eq:linear-equation-property,eq:random-inner-product}では,
  二つのベクトル$a,b\in\F^n$に対して$a=b$かどうかを検証するために, ランダムなベクトル$r\sim\F^n$に対して$r^\top a = r^\top b$かどうかを確認していた.
  では, 二つの行列$A,B\in\F^{n\times n}$に対して$A=B$かどうかを検証するにはどうすればよいだろうか.
  素朴な方法として, 二つのランダムなベクトル$s,r\sim \F^n$を選び, $r^\top A s = r^\top B s$かどうかを確認するという方法が考えられる.
  \begin{exercise}{テンソル積の確認}{check-tensor-product}
    有限体$\F$ (ただし$\abs{\F}=q$) 上の二つの行列$A,B\in\F^{n\times n}$に対して以下が成り立つことを示せ:
    \begin{align*}
      \Pr_{r,s\sim\F^n}\qty[ r^\top A s \ne r^\top B s ] = \begin{cases}
        0 & \text{if } A=B \\
        1-1/q & \text{if } A\ne B.
      \end{cases}
    \end{align*}
  \end{exercise}
  
  \cref{thm:weak-PCP-theorem}の証明では, $A=Y$, $B=x x^\top$であった.
  まず\cref{alg:local-verification-LinEq?}のアイデアを模倣しよう.
  この検証者では, ベクトル$y\in\F^n$をそのまま証明として用いるのではなく, それを係数とする線形関数$\pi\colon x\mapsto x^\top y$を証明として用いるというアイデアであった.
  このアイデアを現在の問題設定に拡張しようとすると,
  二つの関数
  \begin{align*}
    &\pi_Y \colon (r,s)\mapsto r^\top Y s,\\
    &\pi_x \colon r\mapsto r^\top x
  \end{align*}
  を証明として用いることになる.
  つまり, 行列$Y$は二次形式の真理値表$\pi_Y$として保持し,
  ベクトル$x$は線形関数の真理値表$\pi_x$として保持する.
  \cref{alg:local-verification-LinEq?}を拡張すると以下の検証者を得る:
  \begin{algorithm}{}{local-verification-3COL?}
    \begin{enumerate}
      \item 入力として$1^n$を受け取り, オラクルアクセスとして$\pi_Y\colon\F^V\times\F^V\to\F$と$\pi_x\colon\F^V\to\F$を受け取る.
      \item 一様ランダムなベクトル$r,s\sim\F^n$を選択する.
      \item オラクルアクセスを用いて$\pi_Y(r,s)$と$\pi_x(r)$を計算する.
      \item $\pi_Y(r,s)=\pi_x(r)\cdot \pi_x(s)$ならば受理し, そうでなければ拒否する.
    \end{enumerate}
  \end{algorithm}
  
  
  
  
  



